\documentstyle[fix,A4,musictex,multicol,12pt]{article}


\newcommand{\sang}[1]{ \vspace*{8mm} 
                       \begin{center} \LARGE \bf #1 \end{center} 
                       \vspace*{6mm}
\addcontentsline{toc}{section}{\protect{ #1 }}
%\addtocontents{toc}{ #1 }

}


\newcommand{\ikkemelodi}{\vspace*{6mm}}

\newcommand{\blankside}{ \newpage
                         \vspace*{6cm}
                         \centerline{Denne siden er med vilje blank}}

\newcommand{\melodi}[1]{ \begin{center} \small \it {\bf Melodi:} #1 
                         \end{center}
                         \vspace*{6mm} }

\newcommand{\vers}[1]{ \begin{center} \parbox{10cm}{#1} \end{center} }


\newcommand{\bredtvers}[1]{ \begin{center} \parbox{12cm}{#1} \end{center} }


\newcommand{\refreng}[1]{ 
\begin{center}
    \parbox{10cm}{
%      \begin{list}{\it \bf Refreng:}
%        {\setlength{\leftmargin}{1.9cm} 
%          \setlength{\labelwidth}{3.2cm}} 
%      \item  #1 
%     \end{list} 
%      }
{\it \bf Refreng: \\}
#1}
  \end{center} 
  }

\newcommand{\bredtrefreng}[1]{ 
\begin{center}
    \parbox{12cm}{
%      \begin{list}{\it \bf Refreng:}
%        {\setlength{\leftmargin}{1.9cm} 
%          \setlength{\labelwidth}{3.2cm}} 
%      \item  #1 
%     \end{list} 
%      }
{\it \bf Refreng: \\}
#1}
  \end{center} 
  
}


\newcommand{\sprefreng}[1]{ 
\begin{center}{ \hspace*{1cm} \parbox{9cm}{
                              {\it \bf Refreng: \\}
                               #1}}
  \end{center} 
  }



\newcommand{\spvers}[1]{ \hspace*{1cm} \parbox{9cm}{#1} }

\newcommand{\sentrvers}[1]
{ 
  \begin{center} 
    \begin{tabular}{l}
      #1
    \end{tabular}
  \end{center}
  }


\newcommand{\noter}[5]{ 
    \def\nbinstruments{2}
    \nbporteesi=0\relax
    \nbporteesii=1\relax

    \def\freqbarno{9999}
    \generalmeter{\meterfrac {#1}{#2}}

    \debutmorceau
    \autolines{#3}{#4}{100}
    \normal
    \small
  }


\newcommand{\sluttnoter}{
                    \finmorceau \maxlinesinpage=0\relax
                    }


\setlength{\parindent}{0cm}

\setlength{\rightmargin}{\leftmargin}

\setlength{\topsep}{0cm}

\setlength{\partopsep}{0cm}


\begin{document}


\centerline{\huge \bf Sangheftestilen}

\bigskip

Dette er en liten hjelp til dem som v�ger seg p� � lage sanghefter for cyb.

Filene \verb'sangheftestil.tex' og \verb'sangheftestildefinisjoner.tex' inneholder endel
spesialkommandoer for sanghefter: 

\verb.\sang{} \melodi{} \noter{}{}{}{}{} \note{}{} \NOte{}{} \NOTe{}{} 
\samnoteu{}{}{}{} \samnotel{}{}{}{} \sluttnoter \vers{} \spvers{} \sentrvers{}.

Sangheftestilen hetes inn i styrefilen med \verb.
\newcommand{\sang}[1]{ \vspace*{8mm} 
                       \begin{center} \LARGE \bf #1 \end{center} 
                       \vspace*{6mm}
\addcontentsline{toc}{section}{\protect{ #1 }}
%\addtocontents{toc}{ #1 }

}


\newcommand{\ikkemelodi}{\vspace*{6mm}}

\newcommand{\blankside}{ \newpage
                         \vspace*{6cm}
                         \centerline{Denne siden er med vilje blank}}

\newcommand{\melodi}[1]{ \begin{center} \small \it {\bf Melodi:} #1 
                         \end{center}
                         \vspace*{6mm} }

\newcommand{\vers}[1]{ \begin{center} \parbox{10cm}{#1} \end{center} }


\newcommand{\bredtvers}[1]{ \begin{center} \parbox{12cm}{#1} \end{center} }


\newcommand{\refreng}[1]{ 
\begin{center}
    \parbox{10cm}{
%      \begin{list}{\it \bf Refreng:}
%        {\setlength{\leftmargin}{1.9cm} 
%          \setlength{\labelwidth}{3.2cm}} 
%      \item  #1 
%     \end{list} 
%      }
{\it \bf Refreng: \\}
#1}
  \end{center} 
  }

\newcommand{\bredtrefreng}[1]{ 
\begin{center}
    \parbox{12cm}{
%      \begin{list}{\it \bf Refreng:}
%        {\setlength{\leftmargin}{1.9cm} 
%          \setlength{\labelwidth}{3.2cm}} 
%      \item  #1 
%     \end{list} 
%      }
{\it \bf Refreng: \\}
#1}
  \end{center} 
  
}


\newcommand{\sprefreng}[1]{ 
\begin{center}{ \hspace*{1cm} \parbox{9cm}{
                              {\it \bf Refreng: \\}
                               #1}}
  \end{center} 
  }



\newcommand{\spvers}[1]{ \hspace*{1cm} \parbox{9cm}{#1} }

\newcommand{\sentrvers}[1]
{ 
  \begin{center} 
    \begin{tabular}{l}
      #1
    \end{tabular}
  \end{center}
  }


\newcommand{\noter}[5]{ 
    \def\nbinstruments{2}
    \nbporteesi=0\relax
    \nbporteesii=1\relax

    \def\freqbarno{9999}
    \generalmeter{\meterfrac {#1}{#2}}

    \debutmorceau
    \autolines{#3}{#4}{100}
    \normal
    \small
  }


\newcommand{\sluttnoter}{
                    \finmorceau \maxlinesinpage=0\relax
                    }


\setlength{\parindent}{0cm}

\setlength{\rightmargin}{\leftmargin}

\setlength{\topsep}{0cm}

\setlength{\partopsep}{0cm}
. f�r \verb.\begin{document}., og definisjonene hentes inn med \verb.\newcommand{\note}[2]{\notes #2 & #1 \enotes}
\newcommand{\Note}[2]{\Notes #2 & #1 \enotes}
\newcommand{\NOte}[2]{\NOtes #2 & #1 \enotes}
\newcommand{\NOTe}[2]{\NOTes #2 & #1 \enotes}

\newcommand{\samnoteu}[4]
{\NOte{\ibu{1}{#1}{#3}\qh1{#1}\tbu{1}\qh1{#2}}{#4}}

\newcommand{\samnotel}[4]
{\NOte{\ibl{1}{#1}{#3}\qb1{#1}\tbl{1}\qb1{#2}}{#4}}

\newcommand{\dsamnoteu}[4]
{\NOte{\ibbu{1}{#1}{#3}\qh1{#1}\tbu{1}\qh1{#2}}{#4}}
. inne i \verb.\begin{music}. og \verb.\end{music}.
 

\medskip
\centerline{\bf \large Et lite eksempel}
\medskip
\begin{tabular}{ll}
\verb.\begin{music}. & markerer starten p� notene.. ligger i styrefila!\\
\verb.\newcommand{\note}[2]{\notes #2 & #1 \enotes}
\newcommand{\Note}[2]{\Notes #2 & #1 \enotes}
\newcommand{\NOte}[2]{\NOtes #2 & #1 \enotes}
\newcommand{\NOTe}[2]{\NOTes #2 & #1 \enotes}

\newcommand{\samnoteu}[4]
{\NOte{\ibu{1}{#1}{#3}\qh1{#1}\tbu{1}\qh1{#2}}{#4}}

\newcommand{\samnotel}[4]
{\NOte{\ibl{1}{#1}{#3}\qb1{#1}\tbl{1}\qb1{#2}}{#4}}

\newcommand{\dsamnoteu}[4]
{\NOte{\ibbu{1}{#1}{#3}\qh1{#1}\tbu{1}\qh1{#2}}{#4}}
. & henter definisjoner (ogs� i styrefil!)\\
\verb.\sang{Tittel}. &         	angir sangens tittel \\
\verb.\melodi{Melodien}.   &   	angir sangens melodi \\
\verb.\noter{4}{4}{4}{3}{4}. &  denne angir at notene begynner, en rekke variable \\
  &                    	settes og notelinjene lages. 1. og 2. parameter angir\\
  &                    	hvilken takt melodien g�r i, 3. og 4. parameter \\
  &                    	angir henholdsvis hvor mange noter det er mellom\\
  &                  	skillestrekene og hvor mange skillestreker det er \\
       &		pr linje, den femte (skal??) angir antall linjer \\
 & \\
\verb.\note{ \qu e }{Li--}. &dette er en kvartnote (\verb.\qu.) e (se refcard) med teksten\\
 &                        ``Li-'' under\\
\verb.\note{ \hu f }{sa}. & dette er en halvnote (\verb.\hu.) f med ``sa'' under\\
\verb.\note{ \wh g }{gikk}. & dette er en helnote (\verb.\wh.)h med  ``gikk'' under\\
\verb.\note{ \qu {^h} }{til}. & dette er en kvartnote med \# foran...\\
\verb.\barre .                &           	dette angir en skillestrek | \\
\verb.\NOte{\hu i}{sko--}. & dette er en halvnote med st�rre avstand til neste note \\
\verb.\NOte{\hu i}{len}. & samme som forrige... \\
\verb.\samnoteu{j}{i}{-1}{i -- den}. & disse henger sammen, 1. parameter er f�rste note, \\
 & 2. prarameter er andre note, \\
 & 3. parameter er stigetallet (samme = 0) \\
 & 4. parameter er teksten under dem.... \\
\verb.\barre. & nok en skillestrek...\\
\verb.\samnotel{i}{j}{1}{nye  kjo--}. & disse henger ogs� sammen men opp ned,\\
 &  det skyldes l'en p� slutten, \\
 &  dette gjelder ogs� \verb.\ql. (motsatt \verb.\qu.) og \verb.\hl. (motsatt \verb.\hu.)\\
% \verb.\barre.  & en ny skille strek..  \\
%\note{\qu{j}}{ny--}
%\note{\qu{j}}{e}
%\note{\hu{i}}{kjo}
%\note{\qu{h}}{len} 
\verb.\sluttnoter.  &         	n� er det slutt p� notene..... \\
\verb.\end{music}.  & avslutter notene (i styrefila!) \\
\end{tabular}

\newpage

N�r det gjelder versene lages de p� denne m�ten:
\medskip

\verb.\vers{Her kommer et vers av en sang jeg ikke kan\\ Trallla la \\}.

\verb.\refreng{Dette er refrenget tralla lalla la\\}.

\verb.\begin{multicols}{2}.

\verb.\spvers{Dette verset er i en spalte\\
        Men det gjoer ingen ting\\}
.

\verb.\spvers{Dette verset er ogsaa i en spalte\\
        Men det gjoer heller ingen ting\\}
.

\verb.\end{multicols}.

\verb.\sentrvers{Dette verset er sentrert under \\
           spalteversene og tar hensyn til \\
           bredden p� verset}
.

\medskip 

Legg merke til at \verb.\spvers{}. m� forekomme inne i \verb.\begin{multicols}{2}.
\medskip

\centerline{Resultatet av eksemplet kommer her:}
\begin{music}
\newcommand{\note}[2]{\notes #2 & #1 \enotes}
\newcommand{\Note}[2]{\Notes #2 & #1 \enotes}
\newcommand{\NOte}[2]{\NOtes #2 & #1 \enotes}
\newcommand{\NOTe}[2]{\NOTes #2 & #1 \enotes}

\newcommand{\samnoteu}[4]
{\NOte{\ibu{1}{#1}{#3}\qh1{#1}\tbu{1}\qh1{#2}}{#4}}

\newcommand{\samnotel}[4]
{\NOte{\ibl{1}{#1}{#3}\qb1{#1}\tbl{1}\qb1{#2}}{#4}}

\newcommand{\dsamnoteu}[4]
{\NOte{\ibbu{1}{#1}{#3}\qh1{#1}\tbu{1}\qh1{#2}}{#4}}

\sang{Tittel} 
\melodi{Melodien}
\noter{4}{4}{4}{3}{4}
\note{ \qu e }{Li--} 
\note{ \hu f }{sa} 
\note{ \wh g }{gikk}
\note{ \qu {^h} }{til} 
\barre 
\NOte{\hu i}{sko--} 
\NOte{\hu i}{len}
\samnoteu{j}{i}{-1}{i -- den} 
\barre
\samnotel{i}{j}{1}{nye  kjo--}
\sluttnoter
\end{music}

\vers{Her kommer et vers av en sang jeg ikke kan\\ Trallla la \\}

\refreng{Dette er refrenget tralla lalla la\\}

\begin{multicols}{2}

\spvers{Dette verset er i en spalte\\
        Men det gjoer ingen ting\\}

\spvers{Dette verset er ogsaa i en spalte\\
        Men det gjoer heller ingen ting\\}

\end{multicols}

\sentrvers{Dette verset er sentrert under \\
           spalteversene og tar hensyn til \\
           bredden p� verset.}

\medskip

 P� omr�det \verb.tex. finnes en del sanger, det er meningen at disse skal 
 formateres med sangheftekommandoene som vist over....
 eksempel p� sang uten spalter ligger p� \verb'tex/olatveiten.tex'
 eksempel p� sang med spalter og sentrert sistevers ligger p� 
 \verb'tex/bondemannen.tex',
 \verb'tex/lambo.tex' viser litt flere muligheter, inklusive noter (som er gale...)

% sangheftet lages ved � inkludere alle de �nskede sangene i
% styrefila (\verb'sang.tex'), slik som eksemplene..... 
% s� kj�res latex p� \verb'sang.tex'

% filen \verb'eks.DVI' viser er eksempel p� hvordan dette ser ut...

% dette er bare en forel�pig versjon, b�de av sangheftestilen og
% hjelpefila

\medskip

 Skrevet av Jon E. Dahlen <jonda@ifi.uio.no>




\end{document}
