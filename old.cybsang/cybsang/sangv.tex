\documentstyle[norsk,12pt,A4,isolatin,musictex,multicol]{report}


\newcommand{\sang}[1]{ \vspace*{8mm} \begin{center} \LARGE \bf #1 \end{center} 
                       \vspace*{6mm}}

\newcommand{\melodi}[1]{ \begin{center} \small \it {\bf Melodi:} #1 
                         \end{center}
                         \vspace*{6mm} }

\newcommand{\vers}[1]{ \begin{center} \parbox{10cm}{#1} \end{center} }

\newcommand{\refreng}[1]{ 
  \begin{center} 
    \parbox{10cm}{
      \begin{list}{\it \bf Refreng:}
        {\setlength{\leftmargin}{1.9cm} 
          \setlength{\labelwidth}{3.2cm}} 
      \item  #1 
      \end{list} 
      }
  \end{center} 
  }

\newcommand{\spvers}[1]{ \parbox{9cm}{#1} }

\newcommand{\sentrvers}[1]
{ 
  \begin{center} 
    \begin{tabular}{l}
      #1
    \end{tabular}
  \end{center}
  }


\newcommand{\noter}[5]{ 
    \def\nbinstruments{2}
    \nbporteesi=0\relax
    \nbporteesii=1\relax

    \def\freqbarno{9999}
    \generalmeter{\meterfrac {#1}{#2}}

    \debutmorceau
    \autolines{#3}{#4}{#5}
    \normal
    \small
  }


\newcommand{\sluttnoter}{
                    \finmorceau
                    }


\setlength{\parindent}{0cm}

\setlength{\rightmargin}{\leftmargin}

\setlength{\topsep}{0cm}

\setlength{\partopsep}{0cm}

\begin{document}

\begin{music}
\newcommand{\note}[2]{  \NOtes #2 & #1 \enotes   }







%\sang{LAMBO}



\newcommand{\alle}[1]{  \parbox{8cm}{
                              \setlength{\partopsep}{-0.33cm}
                              \begin{list}{\it \bf Alle:}
                                {\setlength{\leftmargin}{2.3cm} 
                                 \setlength{\labelwidth}{3cm}} 
                                 \item  #1 
                              \end{list} 
                            }
                          }

\newcommand{\ene}[1]{  \parbox{8cm}{
                              \setlength{\partopsep}{-0.33cm}
                              \begin{list}{\it \bf Den ene:}
                                {\setlength{\leftmargin}{2.3cm}
                                 \setlength{\labelwidth}{3cm}} 
                                 \item  #1 
                              \end{list} 
                            }
                          }

%\setlength{\columnseprule}{1mm}


%Dette er bare et eksempel....
%\noter{2}{3}{4}{4}{2}
%\note{\qu{e}}{Se} 
%\note{\qu f}{der}
%\note{\qu f}{sit--}
%\note{\qu e}{ter}
%\barre
%\note{\qu h}{en }
%\note{\qu f}{fyl--}
%\note{\qu f}{le}
%%\note{\qu e}{hund}
%\sluttnoter


\begin{multicols}{2}

\spvers{
\alle{	Se der sitter en fyllehund, \\
	Mine herrer Lambo. \\
	Sett nu glasset for din mund, \\
	Mine herrer Lambo. \\
	Se hvordan hver dr�pe vanker \\
	Nedad halsen p� den dranker. \\
	Lambo, Lambo, \\
	Mine herrer Lambo.}
\vspace*{6mm}}

\spvers{ %\vspace*{5mm}
\ene{Jeg mitt glass utdrukket har,}
\alle{Mine herrer Lambo.}
\ene{Se der fins ei dr�pe kvar,}
\alle{mine herrer Lambo.}
\ene{	Som bevis derp� jeg vende \\
	Glasset p� dets annen ende.}
\alle{ 	Lambo, Lambo, \\
	Mine herrer Lambo.}
\vspace*{6mm}}

\end{multicols}


\sentrvers{
\alle{	Han kunne kunsten, \\
	Han var et j�vla fyllesvin. \\
	S� g�r vi til neste mann, og \\
	ser hva han form�r. \\}
\vspace*{6mm}}



\sang{SK�L KAMERATER}
\ikkemelodi
\vers{
Sk�l kamerater, godt humm�r, \\
tidsnok kommer sorgen. \\
Full i g�r, full i dag, \\
ogs� full i morgen. \\
\\
:/: S�nn ska' det v�re, s�nn ska' det v�re, \\
s�nn ska' det v�re s�nn, hei! :/: \\
}







%\sang{�NGLAMARK}

%\def\elemskip=8pt
\noter{3}{4}{3}{3}{3}
\note{\qu c}{Kal-}
\note{\qu e}{la}
\note{\qu g}{den}
\barre
\samnoteu{c}{e}{3}{�ng -la  -}
%\note{\cu c}{�ng-}
%\note{\cu e}{la}
\samnoteu{e}{e}{0}{mar  -  ken}
%\note{\cu e}{mar-}
%\note{\cu e}{ken}
\samnoteu{e}{g}{3}{el  -  ler}
%\note{\cu e}{el-}
%\note{\cu g}{ler}
\barre
\samnoteu{d}{d}{0}{Him - la}
%\note{\cu d}{Him-}
%\note{\cu d}{la}
\samnoteu{d}{^c}{-1}{ jor -den}
%\note{\cu d}{jor-}
%\note{\cu {^c}}{den}
\samnoteu{d}{e}{1}{om du}
%\note{\cu d}{om}
%\note{\cu e}{du}
\xbarre
%\autolines{3}{4}{100}
%\noter{3}{4}{3}{4}{3}
\note{\hup f}{vill,}
\barre
\note{\qu d}{jor-}
\note{\qup d}{den}
\note{\cu e}{vi}
\xbarre
\note{\qu f}{�rv-}
\note{\qu e}{de }
\note{\qu d}{och}
\xbarre
\note{\qu g}{lun-}
\note{\qup g}{den}
\note{\cu f}{den}
\xbarre
\note{\hu f}{gr�-}
\note{\qu f}{na,}
\sluttnoter

\vers{
Kalla den �nglamarken eller himlajorden om du vil,\\ 
jorden vi �rvde och lunden den gr�na, \\
vildrosor och bl�sippor och lindblommor och kamomill, \\
l�t dom f� leva, de er ju s� sk�na! \\
}
\vers{
L�t barnen dansa som �nglar kring l�nn och alm, \\
leka tittut mellan blommande grenar, \\
l�t f�glar leva och sjunga for oss sin psalm, \\
l�t fiskar simma kring bryggor och stenar! \\
}
\vers{
Sluta at utrota skogarnas alla djur! \\
l�t �rnen flyga, l�t r�djuren l�pa! \\
L�t sista elven, som brusar i v�r natur, \\
brusa alltjemt mellan fjellar och gran och fur! \\
}
\vers{
Kalla den �nglamarken eller himlajordn om du vill, \\
jorden vi �rvde och lunden den gr�na, \\
vildrosor och bl�sippor och lindblommor och kamomill, \\
l�t dom f� leva, dom er ju s� sj�na! \\
}


%\sang{BACCHUS}
\ikkemelodi
\vers{
Her er verden her kommer vi, her er alkoholen \\
og vi kjenner dens melodi, dens poesi og trylleri \\
kom la oss synge verdens sang, \\
sangen om alkoholen, \\
den er kjent fra Kapp til Polen, \\
kom la oss synge verdens sang. \\
:/: TA EN DRINK :/: \\
det er det som for oss er "the missing link". \\
Drikk igjen gamle venn, \\
la oss drikke oss helt inn i himmelen. \\
Sa du vann??, nei fy fan, \\
det er det som har drept mange tusen mann. \\
Er jeg full?? - for no tull \\
la oss drikke til vi g�r omkull!! \\
For her er Bacchus, her kommer vi, \\
her er alkoholen, \\
og vi kjenner dens melodi, dens poesi og trylleri, \\
kom, la oss synge verdens sang, \\
sangen om alkoholen, \\
den er kjent fra Kapp til Polen, \\
kom la oss synge verdens sang. \\
}

%\sang{BALLADEN OM KYBERNETIKK -- \newline FAGET OG DETS UT{\OE}VERE}
\melodi{Fremg�r av teksten} 
\vers{
N�r vektalla blir mange og 217 setter inn, \\
da finns det dem som unders�ker studieplanen sin. \\
S� hvis du avskyr kvantetall og passer deg for dem, \\
da b�r kybernetikken bli ditt �ndelige hjem. \\
}
\refreng{
:/: Heisan og hoppsan og fallerallera \\
v�r studietid p� Blindern vil visst ingen ende ta. :/: \\
}
\vers{
S� m� en til � lese, det er mye � forst� \\
selv om en har hukommelse p� over 1000 K. \\
Men Bjerknes klarer biffen hvis kunnskapen har hull, \\
ved � l�re oss om "r�ksestrau" og annet tull. \\
}
\refreng{
Heisan ....... \\
}
\vers{
Nyquist lager looper, Shannon entropi, \\
mens Wiener oppfant Wiener-br�d og samplingsteori. \\
Hvis "binits" korrelleres i en kode som er "Grey", \\
vil spektret bli ergodisk "in a random way". \\
}
\refreng{
Heisan: ...... \\
}
\vers{
P� faglige ekskursjoner er det mye rart � se, \\
og ofte litt "festivitas" for dem som liker det. \\
Noen f�r for mye, s� slipset sitter skakt, \\
men selv i bakrus traller de i hopsa-takt: \\
}
\refreng{
:/: Hikk-san og hoppsan og fyllerallera, \\
v�r studietid p� Blindern vil nok ingen ende ta! :/: \\
 \\
}
\vers{
(Spesielt for generalforsamlinger- \\
synges langsomt og med verdighet!) \\
 \\
Se hovedfagsstudentene er ogs� kommet inn \\
der sitter de og koser seg med pilseflasken sin. \\
De drikker ikke hva-som-helst, det er no' som alle vet; \\
gratispils til feedback-sl�yfens menighet! \\
 \\
}
\refreng{
Hikk-san ...... \\
 \\
}
\vers{
Omsider kom eksamen som vi har venta p�, \\
men oppgave og muntligpensum er nok s� som s�. \\
Sp�rsm�la de  hagler, det er ingen "kj�re mor", \\
men s�! forkynner sensor de h�ytidelige ord: \\
 \\
}
\refreng{
:/: Heisan og hoppsan og fallerallera, \\
din studietid p� Blindern, den tok slutt i dag! :/: \\
 \\
}
Osmund Fiskaa. \\


%\sang{GAMLE VENNER}
\melodi{Glade enke} 
\begin{multicols}{3}
\spvers{
Gamle venner \\
halsen brenner \\
�L - �L - �L. \\
Hviken veske \\
kan oss leske? \\
�L - �L - �L. \\
Er v�r sang enn ikke \\
ren og klar som s�lv, \\
la oss drikke, \\
la oss drikke, \\
�L - �L - �L. \\
}
\spvers{
Hansens enke \\
nektet skjenke \\
�L - �L - �L. \\
Hansen mukket \\
ei, men sukket \\
�L - �L - �L. \\
Nu er Hansen borte, \\
ble spist opp av m�ll. \\
Husk moralen: \\
sjenk gemalen \\
�L - �L - �L. \\
}

\spvers{
La oss feste \\
med det beste \\
�L - �L - �L. \\
Se det skummer \\
kjeller'n rommer \\
�L - �L - �L. \\
Bamsen har sin binne, \\
hoppen har sitt f�ll. \\
Norges s�nner \\
de har t�nner \\
�L - �L - �L. \\
}
\end{multicols}

\sang{GRAVERNE}
\noter{6}{8}{3}{4}{6}
\note{\cu{d}}{}
\xbarre
\note{\qu{g}}{}
\note{\cu{^f}}{}
\note{\qu{e}}{}
\note{\cup{^f} }{}
\xbarre
\note{\qu{g} }{}
\note{\cu{^f}}{}
\note{\qu{e}}{}
\note{\cu{^f}}{}
\xbarre
\note{\qu{g}}{}
\note{\cu{^f}}{}
\note{\qu{e}}{}
\note{\cu{^f}}{}
\xbarre
\note{\qup{g}}{}
\note{\qu{g}\Ilegl 1e\tleg 1 }{}
\note{\cu{h}}{}
\barre
\note{\ql{i}}{}
\note{\cl{i}}{}
\note{\ql{i}}{}
\note{\cl{i}}{}
\xbarre
\note{\ql{j}}{}
\note{\cl{i}}{}
\note{\qu{h}}{}
\note{\cu{g}}{}
\xbarre
\note{\hup{'a}}{}
\xbarre
\note{\qup{h}}{}
\note{\qp}{}
\note{\cl{k}}{}
\barre
\note{\ql{i}}{}
\note{\cl{i}}{}
\note{\ql{i}}{}
\note{\cl{i}}{}
\xbarre
\note{\qu{g}}{}
\note{\cu{g}}{}
\note{\qu{g}}{}
\note{\cu{g}}{}
\xbarre
\note{\qu{g}}{}
\note{\cu{g}}{}
\note{\qu{^f}}{}
\note{\cu{e}}{}
\xbarre
\note{\qup{d}}{}
\note{\qu{d}}{}
\note{\cu{d}}{}
\xbarre
\note{\qu{e}}{}
\note{\cu{e}}{}
\note{\cu{d}}{}
\note{\ds}{}
\hspace{-2mm}\note{\cu{d}}{}
\barre
\note{\qu{e}}{}
\note{\cu{e}}{}
\note{\cu{d}}{}
\note{\ds}{}
\note{\cu{d}}{}
\xbarre
\note{\qu{g}}{}
\note{\cu{g}}{}
\note{\qu{h}}{}
\note{\cu{^f}}{}
\xbarre
\note{\qup{g}}{}
\note{\qu{g}}{}
\note{\ds}{}
%\xbarre
\sluttnoter
\begin{multicols}{2}
\spvers{
Det er en gammel kirkeg�rd \\
hvor gravene st�r tett, \\
men ennu er det plass til mange fler. \\
Vi ser med lengsel p� hver sjel \\
som er blitt blek og trett. \\
Vi vil ha mer, vi vil ha fler, \\
som vi kan grave ned. \\
}
\sprefreng{
Og �l og dram og �l og dram \\
Og �l og dram og �l og dram \\
\\
\\
\\
}
\spvers{
Vi har en gammel Daimler Bentz \\
med silkesort komfort \\
og s�tlig tung kremeringskarakter. \\
Den kj�rer vi i femte gear \\
hjem til v�r kirkeg�rd, \\
med stadig fler, og fler og fler, \\
som vi kan grave ned. \\
}
\sprefreng{
:/: Og �l ... :/: \\
}

\newpage

\spvers{
Ved middagstid vi g�r en tur \\
blandt parkens h�ye tr�r. \\
Det gleder oss � h�re barn som ler. \\
Et lite syrlig drops \\
med arsenikk til hver is�r. \\
Vi byr dem fler og st�r og ser \\
p� at de svelger ned. \\
}
\sprefreng{
:/: Og �l ... :/: \\
}
\spvers{
En tidsinnstillet bombe sprang \\
og flyet styrtet ned. \\
Vi stod der med v�r Daimler like ved. \\
Titanic ble for oss suksess, \\
vi l� med b�t i le, \\
og dukket ned og hentet fler \\
og grov dem alle ned. \\
}
\sprefreng{
:/: Og �l ... :/: \\
}
\spvers{
En stor del av v�rt klientell \\
forspiser seg p� fett. \\
Og intet er mer gledelig enn det. \\
En kiste til en profit�r \\
som ble s� alt for mett, \\
f�r gullbeslag og rosa polster \\
f�r den senkes ned. \\
}
\sprefreng{
:/: Og �l ... :/: \\
}
\spvers{
Et Hiroshima om igjen \\
det venter vi n� p�. \\
Vi ser p� og vi lar det bare skje. \\
Vi skal ikke beklage oss \\
s� lenge vi kan f�, \\
et lite stykke jord igjen \\
som vi f�r grave p�. \\
}
\sprefreng{
:/: Og �l ... :/: \\
}
\end{multicols}


\sang{�L AV RETTE SLAGET}
\melodi{Kjerringa med staven}
\begin{multicols}{2}
\spvers{
�l av rette slaget \\
har vi n� f�tt laget: \\
�tte potter sol og fire merker dugg, \\
b�lgeskum p� toppen, fine, lette fnugg \\
- �l av rette slaget. \\
}% \\
\spvers{
�let v�rt er herlig, \\
drikk det pent og kj�rlig! \\
�tte potter helse, fire merker smil, \\
den som har v�rt t�rst er aldri mer i tvil: \\
�let v�rt er herlig. \\
}%\\
\end{multicols}




\sang{GRAVERNE}
\noter{6}{8}{3}{4}{6}
\note{\cu{d}}{}
\xbarre
\note{\qu{g}}{}
\note{\cu{^f}}{}
\note{\qu{e}}{}
\note{\cup{^f} }{}
\xbarre
\note{\qu{g} }{}
\note{\cu{^f}}{}
\note{\qu{e}}{}
\note{\cu{^f}}{}
\xbarre
\note{\qu{g}}{}
\note{\cu{^f}}{}
\note{\qu{e}}{}
\note{\cu{^f}}{}
\xbarre
\note{\qup{g}}{}
\note{\qu{g}\Ilegl 1e\tleg 1 }{}
\note{\cu{h}}{}
\barre
\note{\ql{i}}{}
\note{\cl{i}}{}
\note{\ql{i}}{}
\note{\cl{i}}{}
\xbarre
\note{\ql{j}}{}
\note{\cl{i}}{}
\note{\qu{h}}{}
\note{\cu{g}}{}
\xbarre
\note{\hup{'a}}{}
\xbarre
\note{\qup{h}}{}
\note{\qp}{}
\note{\cl{k}}{}
\barre
\note{\ql{i}}{}
\note{\cl{i}}{}
\note{\ql{i}}{}
\note{\cl{i}}{}
\xbarre
\note{\qu{g}}{}
\note{\cu{g}}{}
\note{\qu{g}}{}
\note{\cu{g}}{}
\xbarre
\note{\qu{g}}{}
\note{\cu{g}}{}
\note{\qu{^f}}{}
\note{\cu{e}}{}
\xbarre
\note{\qup{d}}{}
\note{\qu{d}}{}
\note{\cu{d}}{}
\xbarre
\note{\qu{e}}{}
\note{\cu{e}}{}
\note{\cu{d}}{}
\note{\ds}{}
\hspace{-2mm}\note{\cu{d}}{}
\barre
\note{\qu{e}}{}
\note{\cu{e}}{}
\note{\cu{d}}{}
\note{\ds}{}
\note{\cu{d}}{}
\xbarre
\note{\qu{g}}{}
\note{\cu{g}}{}
\note{\qu{h}}{}
\note{\cu{^f}}{}
\xbarre
\note{\qup{g}}{}
\note{\qu{g}}{}
\note{\ds}{}
%\xbarre
\sluttnoter
\begin{multicols}{2}
\spvers{
Det er en gammel kirkeg�rd \\
hvor gravene st�r tett, \\
men ennu er det plass til mange fler. \\
Vi ser med lengsel p� hver sjel \\
som er blitt blek og trett. \\
Vi vil ha mer, vi vil ha fler, \\
som vi kan grave ned. \\
}
\sprefreng{
Og �l og dram og �l og dram \\
Og �l og dram og �l og dram \\
\\
\\
\\
}
\spvers{
Vi har en gammel Daimler Bentz \\
med silkesort komfort \\
og s�tlig tung kremeringskarakter. \\
Den kj�rer vi i femte gear \\
hjem til v�r kirkeg�rd, \\
med stadig fler, og fler og fler, \\
som vi kan grave ned. \\
}
\sprefreng{
:/: Og �l ... :/: \\
}

\newpage

\spvers{
Ved middagstid vi g�r en tur \\
blandt parkens h�ye tr�r. \\
Det gleder oss � h�re barn som ler. \\
Et lite syrlig drops \\
med arsenikk til hver is�r. \\
Vi byr dem fler og st�r og ser \\
p� at de svelger ned. \\
}
\sprefreng{
:/: Og �l ... :/: \\
}
\spvers{
En tidsinnstillet bombe sprang \\
og flyet styrtet ned. \\
Vi stod der med v�r Daimler like ved. \\
Titanic ble for oss suksess, \\
vi l� med b�t i le, \\
og dukket ned og hentet fler \\
og grov dem alle ned. \\
}
\sprefreng{
:/: Og �l ... :/: \\
}
\spvers{
En stor del av v�rt klientell \\
forspiser seg p� fett. \\
Og intet er mer gledelig enn det. \\
En kiste til en profit�r \\
som ble s� alt for mett, \\
f�r gullbeslag og rosa polster \\
f�r den senkes ned. \\
}
\sprefreng{
:/: Og �l ... :/: \\
}
\spvers{
Et Hiroshima om igjen \\
det venter vi n� p�. \\
Vi ser p� og vi lar det bare skje. \\
Vi skal ikke beklage oss \\
s� lenge vi kan f�, \\
et lite stykke jord igjen \\
som vi f�r grave p�. \\
}
\sprefreng{
:/: Og �l ... :/: \\
}
\end{multicols}


\sang{�L AV RETTE SLAGET}
\melodi{Kjerringa med staven}
\begin{multicols}{2}
\spvers{
�l av rette slaget \\
har vi n� f�tt laget: \\
�tte potter sol og fire merker dugg, \\
b�lgeskum p� toppen, fine, lette fnugg \\
- �l av rette slaget. \\
}% \\
\spvers{
�let v�rt er herlig, \\
drikk det pent og kj�rlig! \\
�tte potter helse, fire merker smil, \\
den som har v�rt t�rst er aldri mer i tvil: \\
�let v�rt er herlig. \\
}%\\
\end{multicols}




\sang{GRAVERNE}
\noter{6}{8}{3}{4}{6}
\note{\cu{d}}{}
\xbarre
\note{\qu{g}}{}
\note{\cu{^f}}{}
\note{\qu{e}}{}
\note{\cup{^f} }{}
\xbarre
\note{\qu{g} }{}
\note{\cu{^f}}{}
\note{\qu{e}}{}
\note{\cu{^f}}{}
\xbarre
\note{\qu{g}}{}
\note{\cu{^f}}{}
\note{\qu{e}}{}
\note{\cu{^f}}{}
\xbarre
\note{\qup{g}}{}
\note{\qu{g}\Ilegl 1e\tleg 1 }{}
\note{\cu{h}}{}
\barre
\note{\ql{i}}{}
\note{\cl{i}}{}
\note{\ql{i}}{}
\note{\cl{i}}{}
\xbarre
\note{\ql{j}}{}
\note{\cl{i}}{}
\note{\qu{h}}{}
\note{\cu{g}}{}
\xbarre
\note{\hup{'a}}{}
\xbarre
\note{\qup{h}}{}
\note{\qp}{}
\note{\cl{k}}{}
\barre
\note{\ql{i}}{}
\note{\cl{i}}{}
\note{\ql{i}}{}
\note{\cl{i}}{}
\xbarre
\note{\qu{g}}{}
\note{\cu{g}}{}
\note{\qu{g}}{}
\note{\cu{g}}{}
\xbarre
\note{\qu{g}}{}
\note{\cu{g}}{}
\note{\qu{^f}}{}
\note{\cu{e}}{}
\xbarre
\note{\qup{d}}{}
\note{\qu{d}}{}
\note{\cu{d}}{}
\xbarre
\note{\qu{e}}{}
\note{\cu{e}}{}
\note{\cu{d}}{}
\note{\ds}{}
\hspace{-2mm}\note{\cu{d}}{}
\barre
\note{\qu{e}}{}
\note{\cu{e}}{}
\note{\cu{d}}{}
\note{\ds}{}
\note{\cu{d}}{}
\xbarre
\note{\qu{g}}{}
\note{\cu{g}}{}
\note{\qu{h}}{}
\note{\cu{^f}}{}
\xbarre
\note{\qup{g}}{}
\note{\qu{g}}{}
\note{\ds}{}
%\xbarre
\sluttnoter
\begin{multicols}{2}
\spvers{
Det er en gammel kirkeg�rd \\
hvor gravene st�r tett, \\
men ennu er det plass til mange fler. \\
Vi ser med lengsel p� hver sjel \\
som er blitt blek og trett. \\
Vi vil ha mer, vi vil ha fler, \\
som vi kan grave ned. \\
}
\sprefreng{
Og �l og dram og �l og dram \\
Og �l og dram og �l og dram \\
\\
\\
\\
}
\spvers{
Vi har en gammel Daimler Bentz \\
med silkesort komfort \\
og s�tlig tung kremeringskarakter. \\
Den kj�rer vi i femte gear \\
hjem til v�r kirkeg�rd, \\
med stadig fler, og fler og fler, \\
som vi kan grave ned. \\
}
\sprefreng{
:/: Og �l ... :/: \\
}

\newpage

\spvers{
Ved middagstid vi g�r en tur \\
blandt parkens h�ye tr�r. \\
Det gleder oss � h�re barn som ler. \\
Et lite syrlig drops \\
med arsenikk til hver is�r. \\
Vi byr dem fler og st�r og ser \\
p� at de svelger ned. \\
}
\sprefreng{
:/: Og �l ... :/: \\
}
\spvers{
En tidsinnstillet bombe sprang \\
og flyet styrtet ned. \\
Vi stod der med v�r Daimler like ved. \\
Titanic ble for oss suksess, \\
vi l� med b�t i le, \\
og dukket ned og hentet fler \\
og grov dem alle ned. \\
}
\sprefreng{
:/: Og �l ... :/: \\
}
\spvers{
En stor del av v�rt klientell \\
forspiser seg p� fett. \\
Og intet er mer gledelig enn det. \\
En kiste til en profit�r \\
som ble s� alt for mett, \\
f�r gullbeslag og rosa polster \\
f�r den senkes ned. \\
}
\sprefreng{
:/: Og �l ... :/: \\
}
\spvers{
Et Hiroshima om igjen \\
det venter vi n� p�. \\
Vi ser p� og vi lar det bare skje. \\
Vi skal ikke beklage oss \\
s� lenge vi kan f�, \\
et lite stykke jord igjen \\
som vi f�r grave p�. \\
}
\sprefreng{
:/: Og �l ... :/: \\
}
\end{multicols}


\sang{�L AV RETTE SLAGET}
\melodi{Kjerringa med staven}
\begin{multicols}{2}
\spvers{
�l av rette slaget \\
har vi n� f�tt laget: \\
�tte potter sol og fire merker dugg, \\
b�lgeskum p� toppen, fine, lette fnugg \\
- �l av rette slaget. \\
}% \\
\spvers{
�let v�rt er herlig, \\
drikk det pent og kj�rlig! \\
�tte potter helse, fire merker smil, \\
den som har v�rt t�rst er aldri mer i tvil: \\
�let v�rt er herlig. \\
}%\\
\end{multicols}




%\sang{HOVED�EN}

\noter{4}{4}{4}{4}{5}
\small
\samnoteu{d}{_e}{1}{\small Inn - o -}
\note{\qu f}{ver}
\samnoteu{g}{h}{1}{fjor - den}
\note{\qu i}{en}
\barre
\note{\qu f}{sne -}
\note{\qu f}{kke}
\note{\hu f}{gled,}
\barre
\note{\qu d}{akk,}
\note{\qu b}{hvor}
\note{\qu c}{ti -}
\samnoteu{b}{a}{-1}{me - ne}
\barre
\note{\wh b}{flyr!}
%\barre
%\samnoteu{d}{_e}{1}{Ba  --  ken  --}
%\note{\qu f}{om}
%\samnoteu{g}{h}{1}{�  --  sen}
%\note{\qu i}{gikk}
%\barre
%\note{\qu f}{so  --}
%\note{\qu f}{len}
%\note{\hu f}{ned}
%\barre
%\samnoteu{c}{f}{4}{da -- vi}
%\samnoteu{g}{h}{1}{kom til}
%\note{\qu g}{Dy --}
%\note{\qu c}{na}
%\barre
%\note{\wh f}{fyr.}
\sluttnoter
\normalsize

\vers{
Innover fjorden en snekke gled, \\
akk, hvor timene flyr. \\
Bakenom �sen gikk solen ned \\
da vi kom til Dyna fyr. \\
V�ret er s� vakkert sa Johan, \\
skal vi g� i land? \\
S� gikk vi en deilig sommernatt i land p� Hoved�en. \\
Vi fant oss en vakker knatt og satt og s� utover sj�en, \\
sm�fugler sang i busk og kratt s� jeg ble nesten matt. \\
Det hender s� mangt p� Hoved�en en midtsommernatt. \\
}
\vers{
Timevis satt vi der h�nd i h�nd \\
og han sa s� mye pent, \\
om at vi knyttet et elskovsb�nd \\
slik var det iall fall ment. \\
Og jeg var s� ung og dum enda \\
trodde alt han sa. \\
Det var slik en deilig sommernatt p� gamle Hoved�en, \\
og Gud vet hvor lenge slik vi satt og s� utover sj�en. \\
Sm�fugler sang i busk og kratt s� jeg ble rent betatt. \\
Det hender s� mangt p� Hoved�en en midtsommernatt. \\
}
\vers{
Kj�rlighets lykke kan g� i knas \\
elskovsild bli til is. \\
Tenk at v�r herlige lystseilas \\
endte med totalt forlis. \\
Derfor ber og r�der jeg enhver \\
pike fra fjern og n�r. \\
G� aldri en deilig sommernatt i land p� Hoved�en. \\
Sitt aldri forelsket og betatt og se utover sj�en. \\
Sky alt som heter busk og kratt og husk for allting at \\
det hender s� mangt p� Hoved�en en midtsommernatt. \\
}

%\sang{I SAHARASOLENS BRANN}
\melodi{Ned p� Youngstorvets basar}
\noter{}{}{7}{2}{4}
\note{\lfl i}{}
\note{\cup{h}}{I}
\note{\ccu{g}}{Sa-}
\xbarre
\note{\cup{f}}{hara}
\note{\ccu{f}}{sol-}
\note{\cup{f}}{ens}
\note{\ccu{f}}{bra-}
\note{\qu{f}}{nn}
\note{\clp{i}}{kan}
\note{\ccu{h}}{det}
\xbarre
\note{\cup{g}}{hen-}
\note{\ccu{g}}{de}
\note{\cup{g}}{dann}
\note{\ccu{g}}{og}
\note{\qu{g}}{wann}
\note{\cup{g}}{at}
\note{\ccu{f}}{ka-}
\barre
\note{\cup{e}}{me-}
\note{\ccu{e}}{len}
\note{\cup{e}}{kjen-}
\note{\ccu{g}}{ner}
\note{\clp{j}}{�r-}
\note{\ccl{j}}{ken-}
\note{\clp{j}}{t�r-}
\note{\ccl{i}}{sten}
\xbarre
\note{\ql{i}}{ra-}
\note{\hu{h}}{se}
\note{\cup{f}}{og}
\note{\ccu{f}}{da}
\xbarre
\note{\cu{f}}{leg-}
\note{\ccl{k}}{ger}
\note{\clp{k}}{den}
\note{\ccl{k}}{i}
\note{\ql{k}}{vei}
\note{\clp{j}}{i}
\note{\ccl{i}}{en}
\hspace*{-1.3mm}\barre
\note{\cup{h}}{fyk}
\note{\ccl{j}}{og}
\note{\clp{j}}{i}
\note{\ccl{j}}{en}
\note{\ql{j}}{fei}
\note{\clp{i}}{for}
\note{\ccu{h}}{�}
\xbarre
\note{\cup{g}}{dri-}
\note{\ccu{g}}{kke}
\note{\cu{g}}{i}
\note{\ccu{h}}{en}
\note{\clp{i}}{kil-}
\note{\ccl{i}}{de-}
\note{\cup{h}}{rik}
\note{\ccu{g}}{o-}
\xbarre
\note{\qu{g}}{a-}
\note{\hu{f}}{se}
\sluttnoter

\vers{
I Saharasolens brann \\
kan det hende dann og wann \\
at kamelen kjenner �rkent�rsten rase, \\
og da legger den i vei \\
i en fyk og i en fei \\
for � drikke i en kilderik oase. \\
}
\vers{
Vi har heller ingen ro, \\
vi kamelene p� to, \\
n�r vi t�rster slik at tungen v�r er kr�llet. \\
Nei da iler vi som lyn \\
til oasene i by'n \\
som har kilder med det nydeligste �let. \\
}
\vers{
O, du underbare drikk, \\
n�r jeg ser deg for mitt blikk, \\
ja, da frydes jeg i sinnet og i sjelen. \\
Du kan d�ve all min kval, \\
du er sund og du er sval, \\
-du er livet for den t�rstige kamelen. \\
}

%\sang{THE JOTUNHEIMEN JITTERBUG}
\noter{3}{4}{4}{4}{4}
\note{\cu b}{Hu}
\xbarre
\NOte{\ibu{1}{e}{1}\qhp{1}{e}\tbbu{1}\tbu{1}\qh{1}{f}}
{\hspace*{-2mm}hei!  \hspace*{2mm}kor} 
\hspace*{-4mm}
\note{\qu g}{er}
\NOte{\ibu{1}{f}{-1}\qhp{1}{f}\tbbu{1}\tbu{1}\qh{1}{e}}{det - vel} 
\hspace*{-5mm}
\xbarre
\NOte{\ibu{1}{f}{-1}\qhp{1}{f}\tbbu{1}\tbu{1}\qh{1}{^d}}
{\hspace*{-2mm}friskt - og} 
\hspace*{-4mm}
\note{\qu b}{lett}
\NOte{\ibu{1}{d}{-1}\qhp{1}{d}\tbbu{1}\tbu{1}\qh{1}{b}}{upp - p�} 
\hspace*{-5mm}
\xbarre
\note{\qu e}{fjell -}
\note{\qu e}{et}
\NOte{\ibu{1}{d}{-1}\qhp{1}{^d}\tbbu{1}\tbu{1}\qh{1}{b}}{upp - p�} 
\hspace*{-5mm}
\xbarre
\note{\qu e}{fjell -}
\note{\qu e}{et}
\note{\ds}{}
\sluttnoter

\begin{multicols}{2}
\spvers{
Hu hei! kor er det vel friskt og lett \\
upp� fjellet, upp� fjellet! \\
Her humpar bussen s� gj�rma skvett \\
alle stelle over fjellet. \\
Fr� Gjendeosen til vest i Sogn, \\
det tek ein times tid med folkevogn \\
over fjellet, over fjellet. \\
}
\spvers{
Her legg dei veiene tett i tett. \\
Kor det smell'e! �, det skrell'e! \\
Det er eit eventyr rett og slett: \\
elle melle! vekk med fjellet! \\
B�r B�rson r�skar i gras og lyng, \\
og bunadkledde sm� budeier syng: \\
what a feller! what a feller! \\
}
\spvers{
Sj� ljosreklamen p� Semmelfjell. \\
P� eit stelle h�gt i hellet \\
der ligg the high-class Pier Gint Hotel, \\
der dei spelle heile kveldet. \\
"Kom upp, kom upp fr� den tronge dal." \\
�ss ha ein drusteleg dansesal \\
p� hotellet oppi fjellet. \\
}
\spvers{
Fr� dalen stryk b�de fark og fut \\
opp i fjellet, opp i fjellet, \\
og spring som piccolo og grindagut \\
for hotellet oppi fjellet. \\
I resepsjonen sit B�rson sj�l \\
og sel prospektkort av den siste d�l \\
oppi fjellet, oppi fjellet. \\
}
\spvers{
Her er det slutt p� det gamle gr� \\
seterstellet oppi fjellet. \\
Her f�r du cottage cheese og schnitzel og \\
frikadelle a la Telle. \\
Og blir du tyrst drikk du fransk Martel, \\
for vatnet g�r inn i ein krafttunnel \\
gjennom fjellet, gjennom fjellet. \\
}
\spvers{
Kom opp og sj� p� Anitras dans \\
av ei elle-vill mamselle \\
med nylonsl�r og med blomekrans. \\
For ei kjelle! Ah, qu'elle est belle! \\
Og n�r ho kastar det siste plagg \\
i jive and Jotunheimen jitterbug, \\
d�gger'n smelle, for ei snelle! \\
}

\end{multicols}

\sentrvers{
Og n�r s� soli til kvile gjeng \\
attum fjellet, attum fjellet, \\
d� sig du djupt i din mjuke seng \\
p� hotellet oppi fjellet, \\
og k�yrer neste dag frisk og fin \\
i s�te angar av ein dyr bensin \\
heim fr� fjellet, heim fr� fjellet. \\
}



%\sang{KAGGA}
\melodi{Huttetuttetei}

%\noter{}{}{4}{3}{3}
%\note{\lsh m}{}
%\note{\qu d}{}
%\xbarre
%\note{\cu b}{}
%\note{\cu d}{}
%\note{\cu d}{}
%\note{\cu d}{}
%\note{\cu d}{}
%\note{\cu d}{}
%\note{\ds}{}
%\note{\cu d}{}
%\xbarre
%\note{\cu c}{}
%\note{\cu e}{}
%\note{\cu e}{}
%\note{\cu e}{}
%\note{\qup e}{}
%\note{\cu g}{}
%\xbarre
%\note{\cu f}{}
%\note{\cu f}{}
%\note{\cu f}{}
%\note{\cu f}{}
%\note{\cu f}{}
%\note{\cu f}{}
%\note{\ds}{}
%\note{\cu e}{}
%\sluttnoter
\vers{
Og Kagga l� i kjeller'n, og Kagga den ar full \\
s� kom e'lita lerke og hakka seg et hull, \\
og dr�ppene de rullet, til slutt det ble en dam, \\
og lerka s� sitt snitt til � f� en liten dram. \\
}
\refreng{
En dram, s�nn dann og wann, den tar vi alle mann. \\
En sk�l for lita lerke som kaggehullet fann. \\
Den tar vi alle mann! \\
}
\vers{
Og Kagga l� i kjeller'n og Kagga den var full, \\
og lerka gikk til alka og hvisket om et hull. \\
Og alka dro p� smilen og sa: "En dram er go', \\
jeg f�lger deg til kjeller'n, men jeg m� minst ha to." \\
}
\refreng{
En dram ... \\
}
\vers{
Og Kagga l� i kjeller'n og Kagga den var full, \\
og alka gikk til grisen: Jeg vet et lite hull - \\
bli med meg ned i kjeller'n og ta en dram du me'. \\
Ja vel, sa fyllesvinet, men jeg m� minst ha tre. \\
}
\refreng{
En dram ... \\
}
\vers{
Og Kagga l� i kjeller'n og Kagga den var full, \\
og grisen gikk tl b�tta og vr�vlet om et hull. \\
Og fylleb�tta gliste, men sa i parentes: \\
En fylleb�ttes drammer noters m� i snes. \\
}
\refreng{
En dram ... \\
}
\vers{
Og Kagga l� i kjeller'n og var fremdeles full, \\
og dr�pene de rullet ut av det lille hull, \\
sudenten luktet lunta og s� seg varlig om, - \\
s� lettet han p� Kagga, og da ble Kagga tom. \\
}
\refreng{
En dram ... \\
}

%\sang{KANSKJE KOMMER KONGEN}

\noter{4}{4}{4}{4}{5}
\note{\qu d}{Ka  --}
\xbarre
\note{\hl k\Ilegu 1l \tleg 1}{an  --}
%\zcharnote l{\huslur{2.5\noteskip}}
\samnotel{k}{j}{-1}{--  skje}
\samnotel{i}{j}{1}{kom-mer}
\xbarre
\note{\ql k}{kon  --}
\note{\hu g}{gen}
\samnoteu{h}{i}{1}{hit  --  til}
\xbarre
\note{\ql j}{mid  --}
\note{\ql i}{dag}
\note{\qu h}{n�}
\note{\qu g}{i}
\xbarre
\note{\hup {^f}}{dag.}
\barre\zbarre
\samnoteu{d}{d}{0}{vi  --  har}
\xbarre
\note{\qu g}{dek  --}
\note{\qu g}{ket  --}
\note{\ql i}{p�}
\note{\qu g}{per  --}
\barre
\note{\qu h}{ron  --}
\note{\hl j}{gen}
\samnotel{j}{j}{0}{med-god}
\xbarre
\note{\qlp i}{mat}
\note{\cu h}{av}
\note{\qu g}{al  --}
\note{\qu {^f}}{le}
\xbarre
\note{\hup g}{slag}
\sluttnoter


\begin{multicols}{2}
\spvers{
Kanskje kommer kongen \\
hit til middag n� i dag. \\
Vi har dekket p� perrongen \\
med god mat av alle slag. \\
}
\spvers{
Majones og g�selever \\
med r�dbeter og l�k. \\
Det beste skal p� bordet \\
med kongen p� bes�k. \\
}
\spvers{
Kanskje kommer kongen, \\
kanskje ridende til hest. \\
Hesten skal f� knallbonbonen, \\
for da passer hatten best. \\
}
\spvers{
Kalkun panert med hummer, \\
sm� glass med artisjokk. \\
Knutsen skal servere \\
og Ludvigsen er kokk. \\
}
\spvers{
N�r vi kommer til desserten \\
f�r vi drops med karamell. \\
Det er ikke bra for tenner, \\
det er best vi tar den selv. \\
}
\spvers{
Kanskje kommer kongen, \\
i en bil med stram sj�f�r. \\
Han kan bl�se opp ballongen \\
slik sj�f�rer ofte gj�r. \\
}
\spvers{
Men kanskje kommer'n ikke \\
all v�r oppdekning til tross. \\
Det gj�r ikke s� mye, \\
for da blir det mer til oss. \\
}
\spvers{
Kanskje kommer kongen \\
hit til middag n� i dag. \\
Vi har dekket p� perrongen \\
med god mat av alle slag. \\
}


\end{multicols}

%\sang{ET KORN I AKER'N}
\melodi{Bondemannen}

\noter{6}{8}{3}{3}{3}
\note{\cu d}{Et}
\xbarre
\note{\qu g}{korn}
\note{\cl i}{ i}
\note{\cl k}{ak  --}
\note{\cl i}{er'n}
\note{\cu g}{det}
\xbarre
\note{\qu {^f}}{sa} 
\note{\cu h}{til}
\note{\ql j}{mor:}
\note{\cu {^f}}{hva}
\xbarre
\note{\qu g}{blir}
\note{\cl i}{jeg}
\note{\cl k}{n�r}
\note{\cl i}{jeg}
\note{\cu g}{blir}
\xbarre
\note{\qup d}{stor?}
\sluttnoter


\vers{
Et korn i aker'n det sa til mor: \\
Hva blir jeg n�r jeg blir stor? \\
Jeg synes du skal bli en �lporsjon \\
og f�lge v�r tradisjon. \\
For \\
da blir det mange glass �l, \\
ja, da blir det mange, ja, da blir det mange, \\
glass �l, glass �l, glass �l, glass �l, \\
ja da blir det mange glass �l. \\
}
\vers{
S� vil byggkornet gifte seg \\
det humlen traff p� sin vei, \\
og humlen var ung og uerfar'n, \\
falt straks i byggkornets garn. \\
Nu \\
venter vi mange glass �l, \\
ja, venter vi mange, ja, venter vi mange \\
glass �l, glass �l, glass �l, glass �l, \\
nu venter vi mange glass �l. \\
}
\vers{
S� kom til verden en �lkrabat \\
de tappet hundret fat, \\
kom bokk�l, bayer og pilsner�l \\
kom bokk�l, bayer og pils \\
Og \\
s� ble det mange glass �l \\
og alle fikk mange, og alle fikk mange, \\
glass �l, glass �l, glass �l, glass �l, \\
og alle fikk mange glass �l. \\
}

%\sang{KU I TUNNELEN}

\ikkemelodi
\vers{
Se der kommer det en ku i tunnelen \\
sammen med en liten bikkje. \\
Hva gj�r de to her s� sent p� kvelden ? \\
Hvorfor er de sammen tro ? \\
}
\vers{
Er det han-ku eller er det ku-hun ? \\
Er den farlig eller ikke ? \\
Er det sch�fer eller er det buhund ? \\
Vil den logre eller gj� ? \\
}
\vers{
Vil den logre eller gj�re som de gj�r n�r de g�r ut ? \\
� hattstativ, � hattstativ, Knutsen hent en klut. \\
}
\vers{
Se der kommer det en ku i tunnelen, \\
er den r�mling fra et sirkus ? \\
Kanskje jobbet den p� karusellen \\
for den ser litt svimmel ut. \\
}
\vers{
Vil den logre eller gj�re s�nn som hunder ofte gj�r ? \\
� hattstativ, � hattstativ. V�re hjerter bl�r. \\
}
\vers{
Jammen gikk de ikke rett forbi oss, \\
Se, n� g�r de ut der borte. \\
Ja, og muligens kan ingen si oss \\
hva de skulle her de to. \\
}
\vers{
Var det han-ku eller ku-hun ? \\
Var den farlig eller ikke ? \\
Var det sch�fer eller var det buhund ? \\
Hvorfor var de sammen tro ? \\
}
\vers{
Hvorfor var de sammen tro ? \\
}

%\sang{LEVE INFORMATIKKEN}
\melodi{Kalinka}
\noter{2}{4}{2}{4}{2}
\NOTe{\pointdorgue 8\qu h}{Leve}
\xbarre
\NOTe{\qu g}{New}
\samnoteu{e}{f}{1}{ton,-von}
\xbarre
\NOTe{\qu g}{Neu -}
\samnoteu{e}{f}{1}{mann og}
\xbarre
\NOTe{\qu g}{Tur  --}
\samnoteu{f}{e}{-1}{ing og}
\barre
\note{\qu d}{Knuth!}
\samnoteu{h}{h}{0}{Le -- ve}
\xbarre
\samnoteu{g}{f}{-1}{Sh -- an}
\samnoteu{e}{f}{1}{non og}
\xbarre
\note{\qu g}{Wie --}
\samnoteu{e}{f}{-1}{ner og}
\xbarre
\note{\qu g}{Kal --}
\samnoteu{f}{e}{-1}{man og}
\note{\qu d}{Gauss!}
\sluttnoter

\vers{
Leve informatikken \\
Den �kke matte og fysikk \\
Lenker som bandt s� hardt v�rt store fag \\
Hyll, fagidd-joter, v�re ledere! \\
}
\vers{
Leve Newton, von Neumann og Turing og Knuth! \\
Leve Shannon og Wiener og Kalman og Gauss! \\
Leve Dijkstra, McCarty og Hopper og Hoare, \\
Runge-Kutta, Henrici, de Boor, Tsjebysjeff! \\
}
\vers{
Hullkort-maskin for pion�rer,\\
Skrive-terminal som gener�rer,\\
N� vil vi minnes det som en gang var,\\
den gang vi sang i gledesrus:}
\vers{
Leve DEC-10 og Mycro, Nord-10 og Nord-12. \\
Leve Univac og Eniac og Illiac, Balrog! \\
Leve Cyber og Burroughs, Amdahl, IBM, \\
Wegematic og Nusse og CD og MIX! \\
}
\vers{
Puncher og skrivere skranglet \\
Skjermer hadde ikke samme mangler \\
S� la oss hylle terminalene \\
Det er som vi aldri hadde andre enn dem: \\
}
\vers{
Leve Tandberg, Tektronix, Dec-scope, Infoton! \\
Leve Beehive og Silent, Display, Teletype! \\
Leve Alnabru, Diablo, Titt-titt, Linjegods! \\
Leve Tandberg, og Tandberg og Tandberg, Tandberg! \\
}
\vers{N� har vi X-terminaler,\\
Mac-er som synger og taler,\\
Skriker til maskinen, men f�r ikke svar,\\
Hyll, fagidd-jotter, s� maskinene!\\}
\vers{
Leve Sun-er og SPARC-er og DECstation, Mac,\\
Leve Tandberg og Quadra, Atari og Next,\\
Leve Indy og SE og Convex og Vax,\\
og en PC, Toshiba, DEC-20 og Perq\\}
\vers{
Taster i vei p� mitt manus,\\
Word eller Frame eller Cranus,\\
Nettverk' er nede, s� n� gir jeg opp,\\
Disken er full og min tanke tom\\
}
\vers{
Leve Unix og Simula og Emacs og X,\\
Leve \LaTeX og Beta og Emacs og C,\\
Leve QWERTY og WYSIWYG og Emacs og \TeX,\\
og ESCAPE eller mus med et klikk, og klikk-klikk\\}
\vers{
Fest p� informatikken \\
Dosent og student vanker sammen \\
Det er ikke hver dag vi har tid til slikt \\
Men la n� dette bli en tradisjon \\
}
\vers{
Leve Madsen og Hurlen og Lyche og Dahl! \\
Leve Spurkland og Reenskaug og Nygaard og Maus! \\
Leve Winther og Bjerknes og Hisdal, Wang! \\
Leve Hesjedal og Kirkerud og \\
OLE JOHAN !!!!!!! \\
}


%\sang{MELLOM MATTE OG FYSIKKEN}

\melodi{Tyven, tyven}
\noter{2}{4}{4}{3}{2}
\note{\cu d}{Mel  --}
\note{\cu g}{lom}
\note{\cu g}{mat  --}
\note{\cu g}{te}
\xbarre
\note{\cu {^f}}{og}
\note{\cu e}{fys  --}
\note{\cu e}{ikk  --}
\note{\cu e}{en}
\xbarre
\note{\cu d}{Cyb  --}
\note{\cu {^f}}{kor  --}
\note{\cu h}{et}
\note{\cl j}{det}
\xbarre
\note{\cl i}{fant}
\note{\cu h}{seg}
\note{\qu g}{selv}
\sluttnoter

\begin{multicols}{2}
\spvers{
Mellom matte og fysikken \\
Cybkoret det fant seg selv \\
Mange rare sjeler sangen \\
Tenker som sin beste venn. \\
Vi er samlet n�, for � h�re p� \\
Hyllesten vi s� absolutt b�r f�. \\
}
\spvers{
Om du er litt teknologisk \\
Vi har no' � gi til deg. \\
Hvis du liker bits og diskets \\
teletype og Twinings te. \\
{ \it (Lipton viskes i bakgrunnen) }\\
Fagets idioti, ti det dyrker vi \\
Kom n� derfor p�, v�re m�ter sm�. \\
}
\spvers{
Kyber det betyr en "rormann", \\
st�r det i et oppslagsverk. \\
Faget handler om � styre \\
Verden slik vi tror den er. \\
Vi vil styre den, la oss lede den \\
Bruke feedback'en, sl�yfer og Kalman. \\
(La oss f�re den ut i ulykken, \\
bruke feedbacken, sl�yfer og Kalman). \\
}
\spvers{
Formannen han heter Sve-in \\
Cyben er hans hovedfag \\
Han er ikke lenger leder, \\
Vi har f�tt en ny idag. \\
Han vi �nske vil, hell og lykke til \\
H�per han er gla' for da g�r allting bra. \\
}
\end{multicols}
\sentrvers{I et slikt et stort et selskab \\
m� vi ha en �konom. \\
Erik han jo passer kassa \\
s�rger for at alt g�r bra (ha-ha-ha). \\
Deler ut en sum, liten, stor og rund. \\
Kaos og konkurs, �l og Akkevitt. \\
}

%\sang{NU KLINGER IGJENNOM DEN \newline GAMLE STAD}
\ikkemelodi

\bredtvers{
Nu klinger igjennom den gamle stad p� ny en studentersang, \\
og alle mann alle i rekke og rad svinger opp under begerklang, \\
og mens borgerne v�ker i k�ia og h�rer det glade kang-kang, \\
smeller alle mann, alle mann, alle mann, {alle mann, alle mann,} \\
alle mann. \\
}
\bredtrefreng{
Studenter i den gamle stad ta vare p� byens ry! \\
Husk p� at jenter, �l og dram var kjempernes meny. \\
Og faller I alle mann alle, skal det gjalle fra alle mot sky: \\
La'ke byen f� ro, men la den f� merke den er en studenterby. \\
}
\bredtvers{
I denne gamle staden satt s� mangen en konge stor, \\
og hadde nok av �l fra fat og piker ved sitt bord, \\
og de laga b�ljer i gata n�r hjem i fra gilde de for, \\
og nu sitter de alle mann alle i Valhall og traller til oss i kor: \\
}
\bredtrefreng{
Studenter i den gamle stad ....... \\
}
\bredtvers{
P� Elgeseter var det liv i klosteret dag og natt: \\
Der hadde de sin kagge og der hadde de sin skatt. \\
De herjet i Nonnenes gate og rullet og tullet og datt, \\
og nu skuer de fra himmelen ned og griper sin harpe fatt: \\
}
\bredtrefreng{
Studenter i den gamle stad ....... \\
}
\bredtvers{
N�r vi er vandret hen og staden hviler et �yeblikk, \\
da kommer v�re s�nner og tar opp den gamle skikk: \\
En lek mellom muntre buteljer samt aldri s� lit' erotikk. \\
Og s� sitter vi i himmelen og stemmer opp v�r replikk: \\
}
\bredtrefreng{
Studenter i den gamle stad ....... \\
}

%\sang{BONDEMANNEN}
\melodi{Den friske vind og den milde luft}


\noter{6}{8}{3}{3}{3}
\note{\cu d}{Det}
\xbarre
\note{\qu g}{var}
\note{\cl i}{en}
\note{\cl k}{god}
\note{\cl i}{gam - }
\hspace{-4mm}\note{\cu g}{mel}
\hspace{-5mm}\xbarre
\hspace{-3mm}\note{\qu {^f}}{bon -} 
\hspace{-3mm}\note{\cu h}{de - }
\hspace{-3mm}\note{\ql j}{mann}
\hspace{-1mm}\note{\cu {^f}}{som}
\hspace{-4mm}\xbarre
\hspace{-3mm}\samnoteu{g}{g}{0}{sku - lle }
\hspace{-3mm}\note{\cl i}{g�}
\hspace{-3mm}\note{\cl k}{ut'}
\hspace{-3mm}\note{\cl i}{ef - }
\hspace{-3mm}\note{\cu g}{ter}
\hspace{-3mm}\xbarre
\hspace{-3mm}\note{\qup d}{�l,}
\hspace{-3mm}\sluttnoter


\begin{multicols}{2}
\spvers{
Det var en god gammel bondemann \\
som skulle g� ut' efter �l, \\
som skulle g� ut' efter �l, \\
som skulle g� ut' efter �l, \\
efter �l, efter hopsasa, tralalala \\
som skulle g� ut' efter �l. \\
}% \\
\spvers{
Til kona kom der en ung student \\
mens mannen var ut' efter �l, \\
mens mannen var ut' efter �l, \\
mens mannen var ut' efter �l, \\
efter �l, efter hopsasa, tralalala \\
mens mannen var ut' efter �l. \\
}% \\
\spvers{
Han kysset henne p� rosemunn \\
og klappet henne p� kinn, \\
mens mannen var ut' efter �l, \\
mens mannen var ut' efter �l, \\
efter �l, efter hopsasa, tralalala \\
mens mannen var ut' efter �l. \\
}% \\
\spvers{
Men mannen stod bak en d�r og s� \\
hvordan det hele gikk til. \\
Du trodd' jeg var ut' efter �l, \\
Du trodd' jeg var ut' efter �l, \\
efter �l, efter hopsasa, tralalala \\
du trodd' jeg var ut' efter �l. \\
} %\\
\spvers{
Han tok studenten i buksen bak, \\
og ut av d�ren han smed. \\
Og s� gikk han ut' efter �l, \\
og s� gikk han ut' efter �l, \\
efter �l, efter hopsasa, tralalala \\
og s� gikk han ut' efter �l. \\
} %\\
\spvers{
Moralen er ta din kone med \\
n�r du skal g� ut' efter �l, \\
n�r du skal g� ut' efter �l, \\
n�r du skal g� ut' efter �l \\
efter �l, efter hopsasa, tralalala \\
n�r du skal g� ut' efter �l. \\
}% \\
\end{multicols}
\sentrvers{For studenten moralen blir \\
bakom d�ren � se. \\
N�r mannen er ut' efter �l, \\
n�r du tror'n er ut' efter �l, \\
efter �l, efter hoppsasa, tralalala \\
n�r du tror'n er ut' efter �l. \\
}









\end{music}

\end{document}


