\sang{MUSEVISA}
%\melodi{ved samme navn}
\ikkemelodi
\vers{
N�r nettene blir mange og vinduer dukker opp, \\
vi peker vilt og uhemmet med musas lille kropp. \\
Den farer over bordet mens mark�r'n f�lger med, \\
knapp den lett p� hodet og du f�r det til � skje. \\
}
\refreng{
Hei sann og hopp sann, hvor pekte jeg vel n�, \\
der pekte jeg et annet sted, jeg tror det gikk i st�, \\
hei sann og hopp sann, jeg pekte galt igjen, \\
hvor er min gamle teletype og hullkortleseren. \\
}
\vers{
0g vindu'ne blir borte, de ender helt i "b�nn", \\
Jeg leter vilt febrilsk med musa i min klamme h�nd, \\
Jeg lukker alle vindu'ne, s� blir de alle sm�, \\
men hva som finnes inni dem er vansklig � forst�. \\
}
\refreng{Hei sann og hopp sann....\\}
\vers{
Den store Mac-eksperten er ogs� kommet inn, \\
n� sitter hun og koser seg med Quadra-boksen sin, \\
den ha'kke no' ordentlig musedings, de'r no' enhver kan se, \\
Det er jo bare en knapp der det skulle ha v�rt tre. \\
}
\refreng{Hei sann og hopp sann....\\}
\vers{
Det er jo mye bedere, n�r knappene er tre, \\
Da myldrer det kommandoer, du kan knapt f�lge med, \\
Med shift, control, escape og tab, i vakker harmoni \\
pluss museknapp i tillegg blir det ren magi. \\
}
\refreng{Hei sann og hopp sann....\\}
\vers{
Til slutt s� sier Jens (som er i driftsgruppa) som s�: \\
"Det er morrosamt � bruke mus s� lenge den vil g�. \\
N�r musa stopper opp s� m� maskinen av og p�." \\
I mens s� synger alle vi p� muse-visa n�: \\
}
\refreng{Hei sann og hopp sann....\\}

