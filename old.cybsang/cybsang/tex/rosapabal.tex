\sang{ROSA P� BAL}

\noter{3}{4}{3}{4}{3}
\note{\qu c}{Tenk}
\note{\qu e}{at}
\note{\qu g}{jag}
\barre
\note{\ql j}{dan -}
\note{\qlp j}{sar -}
\note{\cl i}{med}
\barre
\note{\ql i}{An}
\note{\qp}{ - }
\note{\qu h}{ders -}
\barre
\note{\hu h}{son}
\note{\cu i}{lil -}
\note{\cl h}{la}
\barre
\note{\hu g}{jag}
\note{\cl j}{lil -}
\note{\cu h}{la}
\barre
\note{\qup g}{jag}
\note{\cu f}{med}
\note{\cu f}{Fri -}
\note{\cu f}{tiof}
\barre
\note{\qu f}{An}
\note{\qp}{ - }
\note{\qu e}{ders -}
\barre
\note{\hu e}{son!}
\note{\qp}{}
\sluttnoter


\sentrvers{
Tenk att jag dansar med Andersson, lilla jag, \\
lilla jag med Fritiof Andersson! \\
Tenk att bli uppbjuden av en s�'n popul�r person! \\
Tenk vilket underbart liv de'ni f�r! \\
Sej mej, hur kenns det att vara charm�r, \\
sj�mann och cowboy, musiker, artist, \\
det kan vel aldrig bli trist. \\
}
\begin{multicols}{2}
\spvers{
Nej, aldrig trist, fr�ken Rosa, \\
har man som er kavaljer, \\
vart jag en steller min kosa, \\
aldrig forglemmer jag er! \\
}
\spvers{
Ni er en s�ngm� fr�n helikons berg, \\
o, fr�ken Rosa, er linje, er ferg- \\
skuldran, profilen med lockarnas krans, \\
�gonens varma glans! \\
}
\spvers{
Tenk, inspirera herr anderson, lilli jag, \\
inspirera Fritiof Andersson! \\
F�r jag kanhenda min egen s�ng, \\
lilla jag, en g�ng? \\
}
\spvers{
"Rosa p� bal", vackert namn eller hur? \\
B�rjan i moll och finalen i dur. \\
N�r blir den ferdig, herr Andersson sej, \\
visan ni diktar till mej? \\
}
\spvers{
Visan om er, fr�ken Rosa, \\
f�r ni i kvell till ert bord. \\
Medan vi tala p� prosa, \\
diktar jag rimmande ord. \\
}
\spvers{
Tyst, ingen s�g att jag kysste er kind. \\
Kenn hur det doftar fr�n parken av lind. \\
Blommande lindar kring m�nbelyst stig - \\
Rosa jag elskar dig! \\
}
\end{multicols}
