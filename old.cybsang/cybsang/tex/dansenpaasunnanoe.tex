\sang{DANSEN P� SUNNAN�}

\ikkemelodi
\vers{
Der g�r en dans p� Sunnan�, der dansar R�nnerdahl \\
med lilla Eva Liljebeck p� pensionatets bal \\
och genom f�nstren str�mmar in fr�n skerg�rdsnatten sval \\
doft av syrener och jasmin i pensinatets sal. \\
}
\vers{
Och lilla Evas arm er rund och freknig hennes hy \\
och r�d som smultron hennes mun och klenningen er ny. \\
-Herr R�nnerdahl, det er ju ni som tar vem ni vil ha \\
bland alla kvinnor jorden runt, det har jag h�rt, haha! \\
}
\vers{
-Att ta er inte min musik, nej fr�ken, men att ge! \\
Jag sl�sar men er end� rik s� lenge jag kan se. \\
-Vad ser ni d�, herr R�nnerdahl, kanske min nya kjol? \\
-Ja, den och kansje n�got mer! tag hit en bra fiol. \\
}
\vers{
Der g�r en dans p� Sunnan� till R�nnerdahls fiol, \\
der dansar v�gor, dansar vind och sn�n som f�ll i fjol. \\
Den virvlar der, der g�r ett brus igenom park och sal \\
och sommarmorgonen st�r ljus och s�derg�ken gal. \\
}
\vers{
Och lilla Eva dansar ut med fenrik Rosenberg \\
och inga frekner syns p� hyn, s� r�d er hennes ferg. \\
Men R�nnerdahl er blek och sk�n och spelar som en gud \\
och svevar i en h�gre rymd der Eva er hans brud. \\
}
