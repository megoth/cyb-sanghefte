\sang{HOVED�EN}

\noter{4}{4}{4}{4}{5}
\small
\samnoteu{d}{_e}{1}{\small Inn - o -}
\note{\qu f}{ver}
\samnoteu{g}{h}{1}{fjor - den}
\note{\qu i}{en}
\barre
\note{\qu f}{sne -}
\note{\qu f}{kke}
\note{\hu f}{gled,}
\barre
\note{\qu d}{akk,}
\note{\qu b}{hvor}
\note{\qu c}{ti -}
\samnoteu{b}{a}{-1}{me - ne}
\barre
\note{\wh b}{flyr!}
%\barre
%\samnoteu{d}{_e}{1}{Ba  --  ken  --}
%\note{\qu f}{om}
%\samnoteu{g}{h}{1}{�  --  sen}
%\note{\qu i}{gikk}
%\barre
%\note{\qu f}{so  --}
%\note{\qu f}{len}
%\note{\hu f}{ned}
%\barre
%\samnoteu{c}{f}{4}{da -- vi}
%\samnoteu{g}{h}{1}{kom til}
%\note{\qu g}{Dy --}
%\note{\qu c}{na}
%\barre
%\note{\wh f}{fyr.}
\sluttnoter
\normalsize

\vers{
Innover fjorden en snekke gled, \\
akk, hvor timene flyr. \\
Bakenom �sen gikk solen ned \\
da vi kom til Dyna fyr. \\
V�ret er s� vakkert sa Johan, \\
skal vi g� i land? \\
S� gikk vi en deilig sommernatt i land p� Hoved�en. \\
Vi fant oss en vakker knatt og satt og s� utover sj�en, \\
sm�fugler sang i busk og kratt s� jeg ble nesten matt. \\
Det hender s� mangt p� Hoved�en en midtsommernatt. \\
}
\vers{
Timevis satt vi der h�nd i h�nd \\
og han sa s� mye pent, \\
om at vi knyttet et elskovsb�nd \\
slik var det iall fall ment. \\
Og jeg var s� ung og dum enda \\
trodde alt han sa. \\
Det var slik en deilig sommernatt p� gamle Hoved�en, \\
og Gud vet hvor lenge slik vi satt og s� utover sj�en. \\
Sm�fugler sang i busk og kratt s� jeg ble rent betatt. \\
Det hender s� mangt p� Hoved�en en midtsommernatt. \\
}
\vers{
Kj�rlighets lykke kan g� i knas \\
elskovsild bli til is. \\
Tenk at v�r herlige lystseilas \\
endte med totalt forlis. \\
Derfor ber og r�der jeg enhver \\
pike fra fjern og n�r. \\
G� aldri en deilig sommernatt i land p� Hoved�en. \\
Sitt aldri forelsket og betatt og se utover sj�en. \\
Sky alt som heter busk og kratt og husk for allting at \\
det hender s� mangt p� Hoved�en en midtsommernatt. \\
}
