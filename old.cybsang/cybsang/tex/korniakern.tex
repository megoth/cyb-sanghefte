\sang{ET KORN I AKER'N}
\melodi{Bondemannen}

\noter{6}{8}{3}{3}{3}
\note{\cu d}{Et}
\xbarre
\note{\qu g}{korn}
\note{\cl i}{ i}
\note{\cl k}{ak  --}
\note{\cl i}{er'n}
\note{\cu g}{det}
\xbarre
\note{\qu {^f}}{sa} 
\note{\cu h}{til}
\note{\ql j}{mor:}
\note{\cu {^f}}{hva}
\xbarre
\note{\qu g}{blir}
\note{\cl i}{jeg}
\note{\cl k}{n�r}
\note{\cl i}{jeg}
\note{\cu g}{blir}
\xbarre
\note{\qup d}{stor?}
\sluttnoter


\vers{
Et korn i aker'n det sa til mor: \\
Hva blir jeg n�r jeg blir stor? \\
Jeg synes du skal bli en �lporsjon \\
og f�lge v�r tradisjon. \\
For \\
da blir det mange glass �l, \\
ja, da blir det mange, ja, da blir det mange, \\
glass �l, glass �l, glass �l, glass �l, \\
ja da blir det mange glass �l. \\
}
\vers{
S� vil byggkornet gifte seg \\
det humlen traff p� sin vei, \\
og humlen var ung og uerfar'n, \\
falt straks i byggkornets garn. \\
Nu \\
venter vi mange glass �l, \\
ja, venter vi mange, ja, venter vi mange \\
glass �l, glass �l, glass �l, glass �l, \\
nu venter vi mange glass �l. \\
}
\vers{
S� kom til verden en �lkrabat \\
de tappet hundret fat, \\
kom bokk�l, bayer og pilsner�l \\
kom bokk�l, bayer og pils \\
Og \\
s� ble det mange glass �l \\
og alle fikk mange, og alle fikk mange, \\
glass �l, glass �l, glass �l, glass �l, \\
og alle fikk mange glass �l. \\
}
