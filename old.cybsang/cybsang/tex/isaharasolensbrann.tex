\sang{I SAHARASOLENS BRANN}
\melodi{Ned p� Youngstorvets basar}
\noter{}{}{7}{2}{4}
\note{\lfl i}{}
\note{\cup{h}}{I}
\note{\ccu{g}}{Sa-}
\xbarre
\note{\cup{f}}{hara}
\note{\ccu{f}}{sol-}
\note{\cup{f}}{ens}
\note{\ccu{f}}{bra-}
\note{\qu{f}}{nn}
\note{\clp{i}}{kan}
\note{\ccu{h}}{det}
\xbarre
\note{\cup{g}}{hen-}
\note{\ccu{g}}{de}
\note{\cup{g}}{dann}
\note{\ccu{g}}{og}
\note{\qu{g}}{wann}
\note{\cup{g}}{at}
\note{\ccu{f}}{ka-}
\barre
\note{\cup{e}}{me-}
\note{\ccu{e}}{len}
\note{\cup{e}}{kjen-}
\note{\ccu{g}}{ner}
\note{\clp{j}}{�r-}
\note{\ccl{j}}{ken-}
\note{\clp{j}}{t�r-}
\note{\ccl{i}}{sten}
\xbarre
\note{\ql{i}}{ra-}
\note{\hu{h}}{se}
\note{\cup{f}}{og}
\note{\ccu{f}}{da}
\xbarre
\note{\cu{f}}{leg-}
\note{\ccl{k}}{ger}
\note{\clp{k}}{den}
\note{\ccl{k}}{i}
\note{\ql{k}}{vei}
\note{\clp{j}}{i}
\note{\ccl{i}}{en}
\hspace*{-1.3mm}\barre
\note{\cup{h}}{fyk}
\note{\ccl{j}}{og}
\note{\clp{j}}{i}
\note{\ccl{j}}{en}
\note{\ql{j}}{fei}
\note{\clp{i}}{for}
\note{\ccu{h}}{�}
\xbarre
\note{\cup{g}}{dri-}
\note{\ccu{g}}{kke}
\note{\cu{g}}{i}
\note{\ccu{h}}{en}
\note{\clp{i}}{kil-}
\note{\ccl{i}}{de-}
\note{\cup{h}}{rik}
\note{\ccu{g}}{o-}
\xbarre
\note{\qu{g}}{a-}
\note{\hu{f}}{se}
\sluttnoter

\vers{
I Saharasolens brann \\
kan det hende dann og wann \\
at kamelen kjenner �rkent�rsten rase, \\
og da legger den i vei \\
i en fyk og i en fei \\
for � drikke i en kilderik oase. \\
}
\vers{
Vi har heller ingen ro, \\
vi kamelene p� to, \\
n�r vi t�rster slik at tungen v�r er kr�llet. \\
Nei da iler vi som lyn \\
til oasene i by'n \\
som har kilder med det nydeligste �let. \\
}
\vers{
O, du underbare drikk, \\
n�r jeg ser deg for mitt blikk, \\
ja, da frydes jeg i sinnet og i sjelen. \\
Du kan d�ve all min kval, \\
du er sund og du er sval, \\
-du er livet for den t�rstige kamelen. \\
}
