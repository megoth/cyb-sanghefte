\sang{NU KLINGER IGJENNOM DEN \newline GAMLE STAD}
\ikkemelodi

\bredtvers{
Nu klinger igjennom den gamle stad p� ny en studentersang, \\
og alle mann alle i rekke og rad svinger opp under begerklang, \\
og mens borgerne v�ker i k�ia og h�rer det glade kang-kang, \\
smeller alle mann, alle mann, alle mann, {alle mann, alle mann,} \\
alle mann. \\
}
\bredtrefreng{
Studenter i den gamle stad ta vare p� byens ry! \\
Husk p� at jenter, �l og dram var kjempernes meny. \\
Og faller I alle mann alle, skal det gjalle fra alle mot sky: \\
La'ke byen f� ro, men la den f� merke den er en studenterby. \\
}
\bredtvers{
I denne gamle staden satt s� mangen en konge stor, \\
og hadde nok av �l fra fat og piker ved sitt bord, \\
og de laga b�ljer i gata n�r hjem i fra gilde de for, \\
og nu sitter de alle mann alle i Valhall og traller til oss i kor: \\
}
\bredtrefreng{
Studenter i den gamle stad ....... \\
}
\bredtvers{
P� Elgeseter var det liv i klosteret dag og natt: \\
Der hadde de sin kagge og der hadde de sin skatt. \\
De herjet i Nonnenes gate og rullet og tullet og datt, \\
og nu skuer de fra himmelen ned og griper sin harpe fatt: \\
}
\bredtrefreng{
Studenter i den gamle stad ....... \\
}
\bredtvers{
N�r vi er vandret hen og staden hviler et �yeblikk, \\
da kommer v�re s�nner og tar opp den gamle skikk: \\
En lek mellom muntre buteljer samt aldri s� lit' erotikk. \\
Og s� sitter vi i himmelen og stemmer opp v�r replikk: \\
}
\bredtrefreng{
Studenter i den gamle stad ....... \\
}
