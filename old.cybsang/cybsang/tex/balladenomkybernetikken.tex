\sang{BALLADEN OM KYBERNETIKK -- \newline FAGET OG DETS UT{\OE}VERE}
\melodi{Fremg�r av teksten} 
\vers{
N�r vektalla blir mange og 217 setter inn, \\
da finns det dem som unders�ker studieplanen sin. \\
S� hvis du avskyr kvantetall og passer deg for dem, \\
da b�r kybernetikken bli ditt �ndelige hjem. \\
}
\refreng{
:/: Heisan og hoppsan og fallerallera \\
v�r studietid p� Blindern vil visst ingen ende ta. :/: \\
}
\vers{
S� m� en til � lese, det er mye � forst� \\
selv om en har hukommelse p� over 1000 K. \\
Men Bjerknes klarer biffen hvis kunnskapen har hull, \\
ved � l�re oss om "r�ksestrau" og annet tull. \\
}
\refreng{
Heisan ....... \\
}
\vers{
Nyquist lager looper, Shannon entropi, \\
mens Wiener oppfant Wiener-br�d og samplingsteori. \\
Hvis "binits" korrelleres i en kode som er "Grey", \\
vil spektret bli ergodisk "in a random way". \\
}
\refreng{
Heisan: ...... \\
}
\vers{
P� faglige ekskursjoner er det mye rart � se, \\
og ofte litt "festivitas" for dem som liker det. \\
Noen f�r for mye, s� slipset sitter skakt, \\
men selv i bakrus traller de i hopsa-takt: \\
}
\refreng{
:/: Hikk-san og hoppsan og fyllerallera, \\
v�r studietid p� Blindern vil nok ingen ende ta! :/: \\
 \\
}
\vers{
(Spesielt for generalforsamlinger- \\
synges langsomt og med verdighet!) \\
 \\
Se hovedfagsstudentene er ogs� kommet inn \\
der sitter de og koser seg med pilseflasken sin. \\
De drikker ikke hva-som-helst, det er no' som alle vet; \\
gratispils til feedback-sl�yfens menighet! \\
 \\
}
\refreng{
Hikk-san ...... \\
 \\
}
\vers{
Omsider kom eksamen som vi har venta p�, \\
men oppgave og muntligpensum er nok s� som s�. \\
Sp�rsm�la de  hagler, det er ingen "kj�re mor", \\
men s�! forkynner sensor de h�ytidelige ord: \\
 \\
}
\refreng{
:/: Heisan og hoppsan og fallerallera, \\
din studietid p� Blindern, den tok slutt i dag! :/: \\
 \\
}
Osmund Fiskaa. \\

