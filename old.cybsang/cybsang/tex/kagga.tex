\sang{KAGGA}
\melodi{Huttetuttetei}

%\noter{}{}{4}{3}{3}
%\note{\lsh m}{}
%\note{\qu d}{}
%\xbarre
%\note{\cu b}{}
%\note{\cu d}{}
%\note{\cu d}{}
%\note{\cu d}{}
%\note{\cu d}{}
%\note{\cu d}{}
%\note{\ds}{}
%\note{\cu d}{}
%\xbarre
%\note{\cu c}{}
%\note{\cu e}{}
%\note{\cu e}{}
%\note{\cu e}{}
%\note{\qup e}{}
%\note{\cu g}{}
%\xbarre
%\note{\cu f}{}
%\note{\cu f}{}
%\note{\cu f}{}
%\note{\cu f}{}
%\note{\cu f}{}
%\note{\cu f}{}
%\note{\ds}{}
%\note{\cu e}{}
%\sluttnoter
\vers{
Og Kagga l� i kjeller'n, og Kagga den ar full \\
s� kom e'lita lerke og hakka seg et hull, \\
og dr�ppene de rullet, til slutt det ble en dam, \\
og lerka s� sitt snitt til � f� en liten dram. \\
}
\refreng{
En dram, s�nn dann og wann, den tar vi alle mann. \\
En sk�l for lita lerke som kaggehullet fann. \\
Den tar vi alle mann! \\
}
\vers{
Og Kagga l� i kjeller'n og Kagga den var full, \\
og lerka gikk til alka og hvisket om et hull. \\
Og alka dro p� smilen og sa: "En dram er go', \\
jeg f�lger deg til kjeller'n, men jeg m� minst ha to." \\
}
\refreng{
En dram ... \\
}
\vers{
Og Kagga l� i kjeller'n og Kagga den var full, \\
og alka gikk til grisen: Jeg vet et lite hull - \\
bli med meg ned i kjeller'n og ta en dram du me'. \\
Ja vel, sa fyllesvinet, men jeg m� minst ha tre. \\
}
\refreng{
En dram ... \\
}
\vers{
Og Kagga l� i kjeller'n og Kagga den var full, \\
og grisen gikk tl b�tta og vr�vlet om et hull. \\
Og fylleb�tta gliste, men sa i parentes: \\
En fylleb�ttes drammer noters m� i snes. \\
}
\refreng{
En dram ... \\
}
\vers{
Og Kagga l� i kjeller'n og var fremdeles full, \\
og dr�pene de rullet ut av det lille hull, \\
sudenten luktet lunta og s� seg varlig om, - \\
s� lettet han p� Kagga, og da ble Kagga tom. \\
}
\refreng{
En dram ... \\
}
