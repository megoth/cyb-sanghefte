\sang{SOPHUS}
\melodi{Sov, Dukkelise}

\noter{2}{4}{3}{4}{4}
\note{\cu c}{So --}
\xbarre
\note{\qu h}{phus}
\note{\cu h}{han}
\note{\cl j}{er}
\xbarre
\note{\ql i}{vel}
\note{\cu h}{et}
\note{\cu g}{eks --}
\xbarre
\note{\qup f}{em --}
\hspace*{-3mm}\note{\cu g}{pl --}
\hspace*{-3mm}\xbarre\hspace*{-3mm}
\note{\qu h}{ar}\hspace*{-3mm}
\note{\ds}{}\hspace*{-3mm}
\note{\cl j}{den}\hspace*{-3mm}
\xbarre\hspace*{-3mm}
\note{\cl k}{bes --}\hspace*{-3mm}
\note{\ql k}{te}\hspace*{-3mm}
\note{\cl i}{venn}\hspace*{-3mm}
\xbarre\hspace*{-3mm}
\note{\qu f}{en}\hspace*{-3mm}
\note{\cl i}{kan}\hspace*{-3mm}
\note{\cl k}{f�}\hspace*{-3mm}
%\barre
%\note{}{}
%\barre
%\note{}{}
%\note{}{}
%\note{}{}
\sluttnoter

\begin{multicols}{2}
\spvers{
Sophus han er vel et eksemplar, \\
den beste venn en kan f�. \\
n�r no'n i min kasse har tatt en sigar, \\
da sier jeg altid som s�: \\
}
\sprefreng{
Sophus har v�rt der \\
saken er klar. \\
Sophus han l�ner alt hva jeg har. \\
Sophus han er n� lik'vel min venn, \\
alt hva han l�ner f�r jeg igjen, \\
tror jeg trall - lall - la. \\
}
\spvers{
Jeg har i min have et p�retre, \\
der hviler jeg n�r jeg er trett, \\
men jeg m� ha v�rt et utrolig fe, \\
en morgen var p�rene ett. \\
}
\sprefreng{Sophus har v�rt der\dots \\}
\spvers{
Med myrtkrans i h�ret og festlig skrud, \\
min kone i kirken har st�tt. \\
Jeg trodde jeg fikk en uskyldig brud, \\
men der var jeg bleven for�dt. \\
}
\sprefreng{Sophus har v�rt der\dots \\}
\spvers{
Jeg var jo s� glad da en s�nn jeg fikk, \\
han var meg s� inderlig kj�r. \\
Men likvel det meg til hjertet gikk, \\
for tanken l� snublende n�r. \\
}
\sprefreng{Sophus har v�rt der\dots \\}
\spvers{
Den gang jeg var d�d jeg til helvede for, \\
der ligger jeg n� p� et brett. \\
Det er ikke sikkert det'r no' jeg tror, \\
det er en som har solgt mitt skjelett. \\
}
\sprefreng{Sophus har v�rt der\dots \\}

\end{multicols}
