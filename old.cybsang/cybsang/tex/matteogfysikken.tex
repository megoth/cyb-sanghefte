\sang{MELLOM MATTE OG FYSIKKEN}

\melodi{Tyven, tyven}
\noter{2}{4}{4}{3}{2}
\note{\cu d}{Mel  --}
\note{\cu g}{lom}
\note{\cu g}{mat  --}
\note{\cu g}{te}
\xbarre
\note{\cu {^f}}{og}
\note{\cu e}{fys  --}
\note{\cu e}{ikk  --}
\note{\cu e}{en}
\xbarre
\note{\cu d}{Cyb  --}
\note{\cu {^f}}{kor  --}
\note{\cu h}{et}
\note{\cl j}{det}
\xbarre
\note{\cl i}{fant}
\note{\cu h}{seg}
\note{\qu g}{selv}
\sluttnoter

\begin{multicols}{2}
\spvers{
Mellom matte og fysikken \\
Cybkoret det fant seg selv \\
Mange rare sjeler sangen \\
Tenker som sin beste venn. \\
Vi er samlet n�, for � h�re p� \\
Hyllesten vi s� absolutt b�r f�. \\
}
\spvers{
Om du er litt teknologisk \\
Vi har no' � gi til deg. \\
Hvis du liker bits og diskets \\
teletype og Twinings te. \\
{ \it (Lipton viskes i bakgrunnen) }\\
Fagets idioti, ti det dyrker vi \\
Kom n� derfor p�, v�re m�ter sm�. \\
}
\spvers{
Kyber det betyr en "rormann", \\
st�r det i et oppslagsverk. \\
Faget handler om � styre \\
Verden slik vi tror den er. \\
Vi vil styre den, la oss lede den \\
Bruke feedback'en, sl�yfer og Kalman. \\
(La oss f�re den ut i ulykken, \\
bruke feedbacken, sl�yfer og Kalman). \\
}
\spvers{
Formannen han heter Sve-in \\
Cyben er hans hovedfag \\
Han er ikke lenger leder, \\
Vi har f�tt en ny idag. \\
Han vi �nske vil, hell og lykke til \\
H�per han er gla' for da g�r allting bra. \\
}
\end{multicols}
\sentrvers{I et slikt et stort et selskab \\
m� vi ha en �konom. \\
Erik han jo passer kassa \\
s�rger for at alt g�r bra (ha-ha-ha). \\
Deler ut en sum, liten, stor og rund. \\
Kaos og konkurs, �l og Akkevitt. \\
}
